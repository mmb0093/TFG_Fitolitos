\apendice{Especificación de Requisitos}

\section{Introducción}
En este anexo se van a especificar y formalizar los objetivos y requisitos del proyecto mediante tablas y diagramas.

\section{Objetivos generales}
Los objetivos globales de este proyecto son:
\begin{itemize}
	\item Confeccionar una herramienta que permita el etiquetado de distintos tipos de fitolitos.
	
	\item Ubicar dicha herramienta en un servidor web y añadir control de acceso.
	
	\item Que se puedan subir imágenes al servidor junto a la información que contengan las etiquetas para poder entrenar un modelo.

\end{itemize}
\section{Catalogo de requisitos}
En función de los objetivos que se han especificado, se va a extraer el conjunto de requisitos, funcionales y no funcionales, para el diseño de la aplicación.

\subsection{Requisitos funcionales}
Se van a recoger los requisitos funcionales~\cite{fun}, que son aquellos que definen algún tipo de función, dentro de la aplicación.

\begin{itemize}
	\item \textbf{RF-1} Confeccionar una herramienta online que permita subir imágenes para ser etiquetadas.
	\begin{itemize}
		\item \textbf{RF-1.1}  Identificarse a través del login de \textit{Google}
		 	\item \textbf{RF-1.2} En el caso de que los datos introducidos sean erróneos, impedir el acceso a la aplicación. 
		 \item \textbf{RF-1.3} En el caso de que los datos sean correctos, permitir el acceso a la aplicación. 
		 \item \textbf{RF-1.4} Una vez dentro de la aplicación, permitir cerrar la sesión.
		 \item \textbf{RF-1.5} Una vez la sesión esté cerrada, que devuelva al usuario a la vista del login.
		\item \textbf{RF-1.6} Escoger imágenes del equipo desde el que se esté trabajando.
		\item \textbf{RF-1.7} Mostrar las imágenes en el etiquetador.
	\end{itemize}
	\item \textbf{RF-2}: Gestionar las etiquetas que se dibujan sobre las imágenes.
	\begin{itemize}
		\item \textbf{RF-2.1} Seleccionar el tipo de fitolito que se va a etiquetar.
		\item \textbf{RF-2.2} Dibujar la etiqueta sobre la imagen.
		\item \textbf{RF-2.3} Mover etiquetas previamente dibujadas.
		\item \textbf{RF-2.4} Redimensionar etiquetas previamente dibujadas.
		\item \textbf{RF-2.5} Borrar etiquetas previamente dibujadas.
	\end{itemize}
	\item \textbf{RF-3}Que se puedan subir imágenes al servidor junto a la información que contengan las etiquetas para poder entrenar un modelo. 
	\begin{itemize}
		\item \textbf{RF-3.1}Guardar las imágenes subidas en el servidor.
			\item \textbf{RF-3.2} Generar la información de las etiquetas en formato csv.
		\item \textbf{RF-3.3} Guardar las etiquetas generadas en el servidor y descargarlas en local.
		\item \textbf{RF-3.4} Mostrar las imágenes en la galería cuando se suban al servidor.
		\item \textbf{RF-3.5} Eliminar del etiquetador aquellas imágenes que se hayan guardado. 
	\end{itemize}

\end{itemize}
\subsection{Requisitos no funcionales}
Se van a recoger los requisitos no funcionales~\cite{nofun}, que son aquellos que especifican los \textit{atributos de calidad}~\cite{atributo}.

\begin{itemize}
	\item \textbf{RNF-1} La aplicación ha de ser fácil de usar e intuitiva para el usuario.
\end{itemize}
\section{Especificación de requisitos}

En esta sección se va a especificar mediante diagramas de casos de uso en notación UML~\cite{uml}, los requisitos anteriormente definidos.

\imagen{uml}{Diagrama UML general de casos de uso de la aplicación.}
\imagen{uml2}{Diagrama UML desglosado del caso de uso 2: \textit{Etiquetar fitolitos}}
En la imagen \ref{fig:uml} podemos ver el diagrama de casos general que vamos a tener. Las tablas que los formalizan son la \ref{tabla:b1}, que es del primer caso de uso, la \ref{tabla:b2}, correspondiente al segundo caso de uso  y la \ref{tabla:b3} que hará referencia al tercer caso de uso.

 En la imagen \ref{fig:uml2} podemos ver extendido el segundo caso de uso del primer diagrama. A este le corresponderán las tablas \ref{tabla:b21} referente caso de uso 2.1, \ref{tabla:b22} referente al caso de uso 2.2 y la \ref{tabla:b23} a la que corresponde el caso de uso 2.3.
 Del caso de uso 2.2, extienden otros tres casos de uso: \ref{tabla:b221} es la tabla de 2.2.1, la \ref{tabla:b222} la del caso de uso 2.2.2 y la del caso de uso 2.2.3 \ref{tabla:b223}.








\tablaSmallSinColores{Caso de uso 1: Subir imágenes }{p{3cm} p{.75cm} p{9.5cm}}{b1}{
	\multicolumn{3}{l}{Caso de uso 1:  Subir imágenes} \\
}
{
	Descripción                            & \multicolumn{2}{p{10.25cm}}{El usuario puede subir imágenes que tenga en su equipo, puede seleccionar una o varias.} \\\hline
	\multirow{2}{3.5cm}{Requisitos}  &\multicolumn{2}{p{10.25cm}}{RF-1} \\\cline{2-3}
	& \multicolumn{2}{p{10.25cm}}{RF-1.1} \\\cline{2-3}
	& \multicolumn{2}{p{10.25cm}}{RF-1.2}  \\\cline{2-3}
	& \multicolumn{2}{p{10.25cm}}{RF-1.3} 
	\\\cline{2-3}
	& \multicolumn{2}{p{10.25cm}}{RF-1.4} 
	\\\cline{2-3}
	& \multicolumn{2}{p{10.25cm}}{RF-1.5} 
	\\\cline{2-3}
	& \multicolumn{2}{p{10.25cm}}{RF-1.6}
	\\\cline{2-3}
	& \multicolumn{2}{p{10.25cm}}{RF-1.7}  
	\\\hline
	Precondiciones                         &   Ninguna   & 
	\\\hline
	\multirow{2}{3.5cm}{Secuencia normal}  & Paso & Acción \\\cline{2-3}
	& 1    & El usuario pulsa el botón para subir imágenes. 
	\\\cline{2-3}
	& 2    & El usuario selecciona una imagen o conjunto de imágenes.
	\\\cline{2-3}
	& 3    & El usuario selecciona abrir imágenes.
	\\\cline{2-3}
	& 4    &  Se cargan la imagen o las imágenes en el etiquetador.                                 	 
	\\\hline
	Postcondiciones                        & \multicolumn{2}{p{10.25cm}}{La imagen se muestra en el canvas del etiquetador.} \\\hline
	Excepciones                        & \multicolumn{2}{p{10.25cm}}{Ninguna}\\\hline
	Importancia                            & Alta \\\hline
	Urgencia                               & Alta \\\hline
	Comentarios                            & & \\
}

\tablaSmallSinColores{Caso de uso 2: Etiquetado de fitolitos }{p{3cm} p{.75cm} p{9.5cm}}{b2}{
	\multicolumn{3}{l}{Caso de uso 2: Etiquetado de fitolitos} \\
}
{
	Descripción                            & \multicolumn{2}{p{10.25cm}}{El usuario puede dibujar, borrar y modificar etiquetas sobre las imágenes que ha subido} \\\hline
	\multirow{2}{3.5cm}{Requisitos}  &\multicolumn{2}{p{10.25cm}}{RF-2}
    \\\cline{2-3}
	& \multicolumn{2}{p{10.25cm}}{RF-2.1} \\\cline{2-3}
	& \multicolumn{2}{p{10.25cm}}{RF-2.2}  \\\cline{2-3}
	& \multicolumn{2}{p{10.25cm}}{RF-2.3}
	\\\cline{2-3}
	& \multicolumn{2}{p{10.25cm}}{RF-2.4}
	\\\cline{2-3}
	& \multicolumn{2}{p{10.25cm}}{RF-2.5} 
	\\\hline
	Precondiciones                         &  \multicolumn{2}{p{10.25cm}}{Que haya imágenes subidas en el etiquetador}
	\\\hline
	\multirow{2}{3.5cm}{Secuencia normal}  & Paso & Acción \\\cline{2-3}
	& 1    & El usuario selecciona una imagen subida.
	\\\cline{2-3}
	& 2    & El usuario selecciona un tipo de etiqueta
	\\\cline{2-3}
	& 3    & El usuario dibuja etiquetas
	\\\cline{2-3}
	& 4    &  El usuario modifica etiquetas.
	\\\cline{2-3}
	& 5    &   El usuario borra etiquetas.                             	 
	\\\hline
	Postcondiciones                        & \multicolumn{2}{p{10.25cm}}{La imagen se muestra en el canvas del etiquetador.} \\\hline
	Excepciones                        & \multicolumn{2}{p{10.25cm}}{No hay cargado en el etiquetador ninguna imagen.}\\\hline
	Importancia                            & Alta \\\hline
	Urgencia                               & Alta \\\hline
	Comentarios                            & & \\
}

\tablaSmallSinColores{Caso de uso 3: Generar datos de entrenamiento }{p{3cm} p{.75cm} p{9.5cm}}{b3}{
	\multicolumn{3}{l}{Caso de uso 3: Generar datos de entrenamiento} \\
}
{
	Descripción                            & \multicolumn{2}{p{10.25cm}}{El usuario puede guardar las imágenes y las etiquetas} \\\hline
	\multirow{2}{3.5cm}{Requisitos}  &\multicolumn{2}{p{10.25cm}}{RF-3}
	\\\cline{2-3}
	& \multicolumn{2}{p{10.25cm}}{RF-3.1} \\\cline{2-3}
	& \multicolumn{2}{p{10.25cm}}{RF-3.2}  \\\cline{2-3}
	& \multicolumn{2}{p{10.25cm}}{RF-3.3}
	\\\cline{2-3}
	& \multicolumn{2}{p{10.25cm}}{RF-3.4}
	\\\cline{2-3}
	& \multicolumn{2}{p{10.25cm}}{RF-3.5} 
	\\\hline
	Precondiciones                         &  Ninguna &
	\\\hline
	\multirow{2}{3.5cm}{Secuencia normal}  & Paso & Acción \\\cline{2-3}
	& 1    & El usuario guarda las etiquetas e imágenes.
	\\\cline{2-3}
	& 2    & el usuario ve dichas imágenes guardadas en la galería.
                            	 
	\\\hline
	Postcondiciones                        & \multicolumn{2}{p{10.25cm}}{Las imágenes desaparecen del etiquetdor.} \\\hline
	Excepciones                        & \multicolumn{2}{p{10.25cm}}{No hay cargado en el etiquetador ninguna imagen.}\\\hline
	Importancia                            & Alta \\\hline
	Urgencia                               & Alta \\\hline
	Comentarios                            & & \\
}

\tablaSmallSinColores{Caso de uso 2.1: Seleccionar imagen}{p{3cm} p{.75cm} p{9.5cm}}{b21}{
	\multicolumn{3}{l}{Caso de uso 2.1: Seleccionar imagen} \\
}
{
	Descripción                            & \multicolumn{2}{p{10.25cm}}{El usuario selecciona una de las imágenes que haya subido}\\\hline
	\multirow{2}{3.5cm}{Requisitos}  &\multicolumn{2}{p{10.25cm}}{RF-2}\\\cline{2-3}
		& \multicolumn{2}{p{10.25cm}}{RF-2.1}
	\\\hline
	Precondiciones                         &  \multicolumn{2}{p{10.25cm}}{Ninguna}
	\\\hline
	\multirow{2}{3.5cm}{Secuencia normal}  & Paso & Acción \\\cline{2-3}
	& 1    & El usuario selecciona una imagen subida.                      	 
	\\\hline
	Postcondiciones                        & \multicolumn{2}{p{10.25cm}}{La imagen se muestra en el canvas del etiquetador.} \\\hline
	Excepciones                        & \multicolumn{2}{p{10.25cm}}{No hay cargado en el etiquetador ninguna imagen.}\\\hline
	Importancia                            & Alta \\\hline
	Urgencia                               & Alta \\\hline
	Comentarios                            & & \\
}

\tablaSmallSinColores{Caso de uso 2.2: Seleccionar etiqueta }{p{3cm} p{.75cm} p{9.5cm}}{b22}{
	\multicolumn{3}{l}{Caso de uso 2.2: Seleccionar etiqueta} \\
}
{
	Descripción                            & \multicolumn{2}{p{10.25cm}}{El usuario selecciona una etiqueta de la botonera} \\\hline
	\multirow{2}{3.5cm}{Requisitos}  &\multicolumn{2}{p{10.25cm}}{RF-2}
	\\\cline{2-3}
	& \multicolumn{2}{p{10.25cm}}{RF-2.1}   
	\\\hline
	Precondiciones                         &  \multicolumn{2}{p{10.25cm}}{Que haya imágenes subidas en el etiquetador}
	\\\hline
	\multirow{2}{3.5cm}{Secuencia normal}  & Paso & Acción 
	\\\cline{2-3}
	& 1    & El usuario selecciona un tipo de etiqueta                 	 
	\\\hline
	Postcondiciones                        & \multicolumn{2}{p{10.25cm}}{Hay una etiqueta seleccionada y será el tipo con el que se etiquete hasta que se cambie.} \\\hline
	Excepciones                        & \multicolumn{2}{p{10.25cm}}{Ninguna}\\\hline
	Importancia                            & Alta \\\hline
	Urgencia                               & Alta \\\hline
	Comentarios                            & & \\
}

\tablaSmallSinColores{Caso de uso 2.3: Guardar etiquetas }{p{3cm} p{.75cm} p{9.5cm}}{b23}{
	\multicolumn{3}{l}{Caso de uso 2.3: Guardar etiquetas} \\
}
{
	Descripción                            & \multicolumn{2}{p{10.25cm}}{Se generan los datos de las etiquetas} \\\hline
	\multirow{2}{3.5cm}{Requisitos}  &\multicolumn{2}{p{10.25cm}}{RF-2}
	\\\cline{2-3}
	& \multicolumn{2}{p{10.25cm}}{RF-2.1} \\\cline{2-3}
	& \multicolumn{2}{p{10.25cm}}{RF-2.2}  \\\cline{2-3}
	& \multicolumn{2}{p{10.25cm}}{RF-2.3}
	\\\cline{2-3}
	& \multicolumn{2}{p{10.25cm}}{RF-2.4}
	\\\cline{2-3}
	& \multicolumn{2}{p{10.25cm}}{RF-2.5} 
	\\\hline
	Precondiciones                         &  \multicolumn{2}{p{10.25cm}}{Que haya imágenes subidas en el etiquetador}
	\\\hline
	\multirow{2}{3.5cm}{Secuencia normal}  & Paso & Acción \\\cline{2-3}
	& 1    & El usuario gestiona las etiquetas y la información se actualiza cuando deja de manipular cada etiqueta.                          	 
	\\\hline
	Postcondiciones                        & \multicolumn{2}{p{10.25cm}}{Ninguna} \\\hline
	Excepciones                        & \multicolumn{2}{p{10.25cm}}{Ninguna}\\\hline
	Importancia                            & Alta \\\hline
	Urgencia                               & Alta \\\hline
	Comentarios                            & & \\
}

\tablaSmallSinColores{Caso de uso 2.2.1: Dibujar etiquetas }{p{3cm} p{.75cm} p{9.5cm}}{b221}{
	\multicolumn{3}{l}{Caso de uso 2.2.1: Dibujar etiquetas } \\
}
{
	Descripción                            & \multicolumn{2}{p{10.25cm}}{Se dibujan las etiquetas sobre las imágenes} \\\hline
	\multirow{2}{3.5cm}{Requisitos}  &\multicolumn{2}{p{10.25cm}}{RF-2.2}
	\\\cline{2-3}
	\\\hline
	Precondiciones                         &  \multicolumn{2}{p{10.25cm}}{Que haya imágenes subidas en el etiquetador}
	\\\hline
	\multirow{2}{3.5cm}{Secuencia normal}  & Paso & Acción \\\cline{2-3}
	& 1    & El usuario dibuja las etiquetas sobre la imagen.                          	 
	\\\hline
	Postcondiciones                        & \multicolumn{2}{p{10.25cm}}{Ninguna} \\\hline
	Excepciones                        & \multicolumn{2}{p{10.25cm}}{Ninguna}\\\hline
	Importancia                            & Alta \\\hline
	Urgencia                               & Alta \\\hline
	Comentarios                            & & \\
}

\tablaSmallSinColores{Caso de uso 2.2.2: Modificar etiquetas }{p{3cm} p{.75cm} p{9.5cm}}{b222}{
	\multicolumn{3}{l}{Caso de uso 2.2.2: Modificar etiquetas } \\
}
{
	Descripción                            & \multicolumn{2}{p{10.25cm}}{Se modifica el tamaño o la posición de las etiquetas} \\\hline
	\multirow{2}{3.5cm}{Requisitos}  &\multicolumn{2}{p{10.25cm}}{RF-2.3}
	\\\cline{2-3}
	& \multicolumn{2}{p{10.25cm}}{RF-2.4}
	\\\hline
	Precondiciones                         &  \multicolumn{2}{p{10.25cm}}{Que haya una etiqueta dibujada previamente}
	\\\hline
	\multirow{2}{3.5cm}{Secuencia normal}  & Paso & Acción \\\cline{2-3}
	& 1    & El usuario selecciona la etiquetas.
	\\\cline{2-3}
	& 2    & el usuario mueve la etiqueta de sitio. 
		\\\cline{2-3}
	& 3    & el usuario redimensiona la etiqueta pinchando en el borde.                         	 
	\\\hline
	Postcondiciones                        & \multicolumn{2}{p{10.25cm}}{La etiqueta se ha movido o cambiado de tamaño} \\\hline
	Excepciones                        & \multicolumn{2}{p{10.25cm}}{Ninguna}\\\hline
	Importancia                            & Media \\\hline
	Urgencia                               & Media \\\hline
	Comentarios                            & & \\
}

\tablaSmallSinColores{Caso de uso 2.2.3: Borrar etiquetas }{p{3cm} p{.75cm} p{9.5cm}}{b223}{
	\multicolumn{3}{l}{Caso de uso 2.2.3: Borrar etiquetas } \\
}
{
	Descripción                            & \multicolumn{2}{p{10.25cm}}{El usuario elimina la etiqueta seleccionada} \\\hline
	\multirow{2}{3.5cm}{Requisitos}  &\multicolumn{2}{p{10.25cm}}{RF-2.2}
	\\\cline{2-3}
	\\\hline
	Precondiciones                         &  \multicolumn{2}{p{10.25cm}}{Que haya una etiqueta dibujada previamente}
	\\\hline
	\multirow{2}{3.5cm}{Secuencia normal}  & Paso & Acción \\\cline{2-3}
	& 1    & El usuario selecciona la etiquetas.
	\\\cline{2-3}
	& 3    & el usuario presiona el botón de suprimir o borrar.                         	 
	\\\hline
	Postcondiciones                        & \multicolumn{2}{p{10.25cm}}{La etiqueta se ha eliminado} \\\hline
	Excepciones                        & \multicolumn{2}{p{10.25cm}}{Ninguna}\\\hline
	Importancia                            & Media \\\hline
	Urgencia                               & Media \\\hline
	Comentarios                            & & \\
}
