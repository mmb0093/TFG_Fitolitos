\capitulo{2}{Objetivos del proyecto}

\section{Objetivos globales}

Como conjunto de proyecto se presentan los siguientes objetivos:

\begin{itemize}
	\item Crear un etiquetador que genere un conjunto de entrenamiento recogiendo la información necesaria y estructurándola de forma funcional para su posterior lectura, interpretación y manipulación por parte de un modelo.
	
	\item Desplegar la aplicación en un servidor de producción desde el que se pueda trabajar cómodamente y generando el mínimo conflicto entre los usuarios que etiquetan y los que entrenan el modelo. Se realizará un estudio comparativo entre las distintas opciones de despliegue que han sido probadas.
	
	\item Conocer y comprender el funcionamiento de librerías para realizar el aprendizaje automático como \textit{Tensorflow} y su relación con CUDA.
	
	\item Probar y estudiar diferentes prototipos para la futura detección y clasificación de fitolitos.
\end{itemize}

\section{Objetivos técnicos}

El uso de distintos elementos técnicos tiene como objetivo dentro de este proyecto los siguientes puntos:

\begin{itemize}
	\item Utilizar \textit{HTML} y \textit{Javascript} para elaborar la parte del cliente, en este caso, los elementos que conforman el entorno web del proyecto.
	
	\item El uso de \textit{Python} y de \textit{Flask} para operar en la parte del servidor. \textit{Flask} va a actuar como un intermediario entre las partes de la lógica escrita en \textit{Python} y la parte del cliente explicada en el punto anterior.
	
	\item Generar un control de acceso de usuarios a la aplicación mediante un login. Se utilizará un acceso gestionado por Google por medio de su API \footnote{Enlace a APIs y servicions de google: \url{https://console.developers.google.com/apis/dashboard?project=corded-cortex-201715}}.
	
	\item Despliegue en \textit{Pythonanywhere}\footnote{Página de \textit{Pythonanywhere} \url{https://www.pythonanywhere.com/}}, para el cual se ha necesita conocimientos de \textit{git}. Este tipo de servicio solo permite hostear aplicaciones escritas en \textit{Python}.
	
	\item Proporcionar un servidor seguro que permita una interacción transparente por parte del usuario. Básicamente que el usuario no tenga que instalarse. Que todo los que se haga.
	
	\item Uso de control de versiones mediante \textit{git} y y un servicio central, en este caso \textit{GitHub}.
	
	\item Se va a seguir la metodología de Scrum, haciendo reuniones semanales correspondiente a los sprints que se planteen. 
	
	\item La organización de las \textit{issues} dentro del proyecto se hará mediante \textit{ZenHub} que nos proporcionará un tablero para ordenar las actividades en función de su estado y prioridad.
	
\end{itemize}