\apendice{Documentación técnica de programación}

\section{Introducción}
En este apéndice se van a explicar los requisitos y directorios del proyecto y la documentación para que otra persona retome el desarrollo.
\section{Estructura de directorios}
Las estructura de carpetas del proyecto se encuentra organizada en forma de árbol. En este apartado se explicarán brevemente.

\begin{itemize}
	\item \textbf{Directorio raíz}: En este directorio tenemos el \textit{.gitignore}, el \textit{README} y la licencia del proyecto.
	\begin{itemize}
		\item \textbf{code}: contiene los scripts con la mayor parte de la lógica de la aplicación.
			\begin{itemize}
				\item \textbf{notebook}: ontiene el \textit{notebook} con las comprobaciones de las etiquetas del proyecto anterior.
				\item \textbf{pytolithclassifier}: en este directorio podemos encontrar el fichero \textit{boxfile.yml}, uno de los más importantes debido a que es necesario para el despliegue de la aplicación. También se encuentra el \textit{Procfile}, \textit{requirements.txt} con las librerías y las versiones que hay que instlar. El fichero \textit{run.py}, el cual hay que ejecutar para poner en marcha la aplicación y el \textit{runtime.txt}.
				\begin{itemize}
					\item \textbf{.idea}: Archivos referentes al delpliegue.
					\item \textbf{app}: Contiene los ficheros en \textit{HTML, JavaScript y CSS} además de \textit{views.py}, que va a tener la librería de \textit{Flask} en uso y \textit{init.py} que se encarga de pasar los datos de ejecución al \textit{run.py}.
					
						\subitem \textbf{images}: contiene las imágenes que se han guardado en local.
						\subitem \textbf{static}: contiene dos carpetas llamadas \textit{js} y \textit{ css} y contienen los archivos en \textit{JavaScript} y \textit{css} respectivamente.
						\subitem \textbf{templates}: contiene los ficheros \textit{HTML} 
					
					\item \textbf{etc}: contiene los ficheros \textit{gunicorn.py} y \textit{nginx.conf}, los cuales se usan en el despliegue.
					\item \textbf{recursos del clasificador}: Contiene aquellos recursos que se pudieron conseguir en el avance con el clasificador, como los \textit{csv} de entrenamiento, los scripts de conversión, el mapa de etiquetas y diversos archivos de configuración.
				\end{itemize}
			\end{itemize}
		\item \textbf{Docs}: Contiene la documentación del proyecto, anexos y memoria.
		\begin{itemize}
			\item \textbf{img}: contiene las imágenes empleadas en la documentación.
			\item \textbf{tex}: contiene los diferentes apartados de anexos y memoria.
		\end{itemize}
		\item \textbf{Lecturas y enlaces}: contiene un fichero escrito en \textit{marckdown}\footnote{Está hecho en \textit{marckdown} porque se editaba desdeel repositorio de \textit{GitHub}. Para saber más de \textit{marckdown}: \url{https://es.wikipedia.org/wiki/Markdown}}
	\end{itemize}
\end{itemize}
\section{Manual del programador}
En este apartado se van a introducir aspectos importantes a tener en cuenta dentro de la continuación con el desarrollo.

\subsection{Librerías usadas}
Para facilitar la labor de instalar librerías se ha facilitado dentro de la carpeta de \textit{pytolithclassifier} un documento \textit{txt} con todas las librerías con las versiones especificadas. Si bien es cierto que es mejor trabajar con las librerías actualizadas, sería recomendable probar primero el proyecto con las versiones recomendadas y después ya pensar en actualizarlas.

\subsection{\textit{Flask}}
Esta librería es la más importante que se ha usado, ya que es la piedra angular que encaja la lógica escrita en \textit{Python} con la interfaz escrita en \textit{HTML, JavaScript y css}.

\subsection{CUDA y \textit{Tensorflow}}
Antes de intentar hacer nada del clasificador, es necesario aclarar que se van a necesitar tanto CUDA, para el procesamiento de imágenes, como \textit{Tensorflow} para entrenar al modelo.

El problema está en que las versiones más nuevas de ambas son incompatibles\footnote{Hay muchos foros que tratan esta temática, pero en este hilo está muy bien explicado: \url{https://stackoverflow.com/questions/50622525/which-tensorflow-and-cuda-version-combinations-are-compatible}}.

\subsection{Clasificador}
Recomiendo seguir este tutorial~\cite{git} por varias razones.

La primera razón es que, si no tienes ningún problema, la cantidad de trabajo requerida es poca, con lo cual, podrás aprender a crear un clasificador de imágenes en un tiempo reducido.

Otra razón es que viene muy bien explicado, el instructor se preocupa porque cada explicación sea minuciosa, no hace algo sin contar por qué.

Otra razón es que te deja disponible todos los recursos en un su repositorio y puedes seguir el tutorial por vídeo si así lo deseas, no obstante, recomiendo seguir el tutorial a través del repositorio, ya que se explican más cosas.

Muchos de los recursos necesarios para este tutorial ya están hechos y guardados en la carpeta de \textit{recursos del clasificador}, lo cual puede o bien servir de ejemplo o ser usados.
\section{Compilación, instalación y ejecución del proyecto}
Si se quiere ejecutar en local, una vez se tenga bien adecuado el entorno de \textit{Python} hay que ejecutar el fichero \textit{run.py}, el cual se encuentra dentro de la carpeta \textit{code/pytolithclassifier}.

No hay nada que instalar más allá de tener in IDE para \textit{Python} y un entorno de \textit{Python} adecuado, preferiblemente \textit{Anaconda}.

El enlace de la página de la aplicación se encuentra en el repositorio, solo hay que pincharle para acceder.

\section{Pruebas del sistema}
Apenas se han realizado pruebas a lo largo del proyecto debido a la falta de tiempo. Realmente cuando se está desarrollando un producto de software es decuado realizar pruebas desde los primeros pasos del desarrollo.

Sí que se ha puesto a prueba el etiquetador con diferentes imágenes de distintos tamaños para comprobar que escalaba bien.

Se hicieron pruebas en la subida de imágenes, para comprobar que solo se subían determinados formatos.

Sería interesante plantear las pruebas en la siguiente parte del proyecto y utilizar alguna herramienta que lo automatice como \textit{Selenium}\footnote{Para la automatización de teses en páginas web: \url{https://www.seleniumhq.org/}}.