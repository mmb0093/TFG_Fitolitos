\apendice{Documentación de usuario}

\section{Introducción}
En este anexo se explicará de forma detalada como ejecutar la aplicación desde el punto de vista de un usuario medio.
\section{Requisitos de usuarios}
El único requisito que hay que tener en cuenta es que se necesita conexión a internet para ejecutar la aplicación.
\section{Instalación}
No precisa de instalación alguna\footnote{Es una de las razones por las que se decidió hacer online, porque no se necesita tener nada instalado y es más fácil para el usuario}, es online.

Se puede acceder a la aplicación desde aquí: \url{
	http://martamonjeblanco.pythonanywhere.com/}
\section{Manual del usuario}

En este apartado se van a explicar los pasos a seguir para usar la aplicación.

\subsection{Identificación de usuario}
Antes de entrar a la aplicación, se e solicitará al usuario acceder a través de una cuenta de \textit{Google} tal y como se muestra en la imagen \ref{fig:login}.
\imagen{login}{Identificación de usuarios po medio de \textit{Google}}

\subsection{Navegación}
Para cambiar de una vista a otra solo hay que seleccionar a que vista se quiere acceder.

La barra de navegación aparece en la parte superior de la pantalla y se ve como en la imagen \ref{fig:navar}
\imagen{navar}{Barra de navegación de la aplicacón}

\subsection{Inicio}
La pantalla de inicio se verá así \ref{fig:inicio}.

En ella el usuario podrá cerrar sesión como se muestra en la imagen \ref{fig:logout}

Viene una pequeña descripción de la aplicación.

\imagen{logout}{Opción de cerrar sesión en el caso de que se desee salir o cambiar de cuenta.}

\subsection{Etiquetador}
Cuando se accede al etiquetador la vista que se tiene de él es como la de la imagen \ref{fig:vacio}.

\subsubsection{Buscar imágenes}
Hay que pinchar sobre el botón de \textit{Buscar...} ubicado en la parte superior a la izquierda (ver imagen \ref{fig:buscar}).
\imagen{buscar}{Botón de buscar imágenes en el equipo.}

\subsubsection{Seleccionar imágenes}
Cuando se pincha en el botón \textit{Buscar...}, se abre una ventana mostrando los archivos de tipo imagen que tenemos en nuestro dispositivo tal y como se muestra en la imagen \ref{fig:seleccionar}. \imagen{seleccionar}{Seleccionar la imagen o el conjunto de imágenes que se desee.}

Solo se pueden seleccionar archivos de tipo imagen, para evitar posibles errores.
\imagen{vacio}{Etiquetador sin imágenes cargadas}

Una vez estén elegidas as imágenes, hay que darle al botón de \textit{Abrir} y automáticamente cargará las imágenes en la aplicación y cerrara la ventana de selección de imágenes.
\subsubsection{Imágenes cargadas}
Las imágenes cargadas se van a mostrar en el lateral izquierdo como se ve en la imagen \ref{fig:imgguardadas}
\imagen{imgguardadas}{Imágenes que se han cargado en la aplicación pero que aún no se han subido}
\subsubsection{Botones de etiquetas}

Antes de ponerse a etiquetar hay que seleccionar el tipo de fitolito en la botonera (ver imagen \ref{fig:botones}) de la parte superior.
\imagen{botones}{Botonera para seleccionar el tipo de fitolito que se va a etiquetar.}

Si no seleccionase ninguno se establecería el \textit{bilobate} como el fitolito por defecto.

\subsubsection{Dibujar etiquetas}
Para dibujar etiquetas tenemos que tener una imagen cargada en el canvas. Se pincha y se arrastra para generar una etiqueta restangular sobre la imagen y se le asignará el tipo de fitolito que se haya elegido.

\subsubsection{Modificar etiquetas}
Para modificar una etiqueta podemos cambiarla de sitio dentro de la imagen o redimensionarla.

Para cambiarla de sitio hay que seleccionar la imagen, como se ve en la imagen \ref{fig:seleccionada} y pinchar sin soltar hasta que se esté conforme con la nueva ubicación
\imagen{seleccionada}{Etiqueta seleccionada}

Si se quisiera redimensionar, se tendría que seleccionar la etiqueta y pinchar en el borde o esquina que se quiera cambiar.

\subsubsection{Copiar}
Si se desea copia una etiqueta, con el tipo de fitolito incluido, solo hay que seleccionar una etiqueta y hacer \textit{ctrl + c}
\textit{ctrl + v} ya ya se tiene una copia.
\subsubsection{Eliminar etiquetas}
Para elimirar las etiquetas que no se quieran en la imagen o porque se ha cometido un error, Hay que seleccionar la etiqueta y darle al botón de \textit{suprimir} o al de \textit{borrar}.
\subsubsection{Guardar etiquetas}
Una vez se ha acabado de etiquetar las imágenes podemos guardar las etiquetas y las imágenes para su posterior uso.

Hay que pulsar el botón de \textit{Guardar imágenes y etiquetas} que se ve como se muestra en la imagen \ref{fig:guardar}.
\imagen{guardar}{Botón de guardado de imágenes y etiquetas.}

Se guardaran las etiquetas en descargas.
\subsection{Galería}

La opción de la galería es meramente informativa. En ella se muestran las imágenes que se muestran las imágenes que se han subido al servidor como se ve en la imagen \ref{fig:galeria}
\imagen{galeria}{Vista de la galería con las imágenes del servidor}
\subsubsection{title}