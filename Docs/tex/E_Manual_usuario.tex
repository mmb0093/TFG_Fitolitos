\apendice{Documentación de usuario}

\section{Introducción}

\section{Requisitos de usuarios}

\section{Instalación}

\section{Manual del usuario}
\subsection{\textit{Alp’s Labeling Tool (ALT)}}

Para hacer el etiquetado se va a usar \textit{Alp’s Labeling Tool (ALT)}.\\
Se trata de un etiquetador bastante sencillo, tanto de descargar como de usar.\\

\subsubsection{Software necesario}
Para poder hacer uso de ella para la funcionalidad que queremos, en este caso etiquetar imágenes, vamos a neceseitar 3 tipos de software:\\
\begin{itemize}
	\item \href{http://fiji.sc/#download}{Fiji}. Esta herramienta no precisa de instalación como tal, simplemente hay que arrastras la carpeta que contiene la aplicación y ubicarla donde nos resulte más cómoda trabajar on ella.
	\item \href{https://github.com/imagejan/ActionBar/releases/download/SciJava-Parameters/action_bar-2.0.5-SNAPSHOT.jar}{Action Bar}. Se trata de un plugin que hay que instalar directamente en la aplicación Fiji.
	\item \href{https://www.dropbox.com/s/ihkr0ahhif3csvp/ALT_Windows_22mar2017.zip?dl=0}{ALT}. Este macro plugin nos va ha permitir cargar las imágenes, dibujar etiquetas sobre ellas, poner nombres a dichas etiquetas, guardar la información, volver a cargarla y devolver las coordenadas en formato .txt o .csv.
\end{itemize}
\subsubsection{Instalación}

\subsubsection{Uso de la aplicación}
\subsubsection{Recomendaciones}