\capitulo{4}{Técnicas y herramientas}

\section{Etiquetador}


\begin{itemize}
	\item \href{https://sweppner.github.io/labeld/}{LabelID}:Escritorio
	\item \href{http://www.cvlibs.net/software/liblabel/}{LibLabel}: 3D, matlab, escritorio (?)
	\item \href{https://cvhci.anthropomatik.kit.edu/~baeuml/projects/a-universal-labeling-tool-for-computer-vision-sloth/}{Sloth}
	\item \href{https://github.com/tzutalin/labelImg}{LabelImg}
\end{itemize}

\section{Despliegue}
\subsection{Framework : \textit{Flask}}
Para desarrollar una aplicación web necesitamos antes de nada un framework que nos provea de las utilidades propias y un esquema de trabajo. Existen diferentes alternativas, en este caso desarrollados sobre Python, como podrían ser \textit{Flask}, \textit{Pyramid} o \textit{Django}. Fuera de contexto ninguna es mejor que otra, pero desde el punto de vista de la aplicación y el mío propio es adecuado hacer un pequeño análisis de elección.
La decisión está entre\textit{ Django} y \textit{Flask}, por la documentación y los tutoriales existentes, no obstante, aunque \textit{Django} sea más completo, también es menos flexible. Por otro lado \textit{Flask} te permite instalar diferentes utilidades a medida que las vayas necesitando, además de que aunque tiene una documentación más reducida que \textit{Django}, esto la hace más práctica y útil dado que es lo suficientemente completa. \textit{Flask} también es más rápida que \textit{Django} y más sencilla de aprender, por lo tanto emplearemos \textit{Flask}\footnote{Ver documentación de Flask : \url{http://flask.pocoo.org/}}.
\subsection{Nanobox}
\textit{Nanobox}\footnote{Documentación de Nanobox : \url{https://docs.nanobox.io/}} es un servicio que emplea Virtual Box y Docker para crear entornos de desarrollo virtuales dentro de la máquina local. Se configura a través de un fichero yaml llamado \textit{boxfile.yml}, en el cual especificaremos la configuración de los recursos que vayamos a necesitar tanto localmente como en producción. 
