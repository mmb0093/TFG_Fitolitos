\capitulo{4}{Técnicas y herramientas}

\section{Etiquetador}


\begin{itemize}
	\item \href{https://sweppner.github.io/labeld/}{LabelID}:Escritorio
	\item \href{http://www.cvlibs.net/software/liblabel/}{LibLabel}: 3D, matlab, escritorio (?)
	\item \href{https://cvhci.anthropomatik.kit.edu/~baeuml/projects/a-universal-labeling-tool-for-computer-vision-sloth/}{Sloth}
	\item \href{https://github.com/tzutalin/labelImg}{LabelImg}
\end{itemize}

\section{Despliegue}
\subsection{Estructura}
Para desarrollar una aplicación web necesitamos antes de nada un framework que nos provea de las utilidades propias y un esquema de trabajo. Existen diferentes alternativas, en este caso desarrollados sobre python, como podrían ser Flask, Pyramid o Django. Descontestualizadas ninguna es mejor que otra, pero desde el punto de vista de la aplicación y el mío propio es adecuado hacer un pequeño análisis de elección.
La decisión está entre Django y Flask, por la documentación y los tutoriales existentes, no obstante, aunque Django sea más completo, también es menos flexible. Por otro lado Flask te permite instalar diferentes utilidades a medida que las vayas necesitando, además de que aunque tiene una documentación más reducida que Django, esto la hace más práctica y útil dado que es lo suficientemente completa. Flask también es más rápida que Django y más sencilla de aprender, por lo tanto emplearemos Flask.
\subsection{Servidor}