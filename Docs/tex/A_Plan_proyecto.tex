\apendice{Plan de Proyecto Software}

\section{Introducción}

\section{Planificación temporal}
\subsection{Sprint 1}
En este sprint se han cubierto las primeras tareas del proyecto. En este caso las tareas han estado relacionadas principalmente con la lectura y comprensión de la documentación previa del mismo trabajo.
También se comenzó con la búsqueda de etiquetadores, para encontrar el más adecuado para el uso que le queremos dar.
\subsection{Sprint 2}
En este segundo sprint la actividad a estado centrada en la elección del etiquetador y en el aprendizaje del mismo. Finalmente se tomó la decisión de utilizar \textit{Alp’s Labeling Tool (ALT)}. 
Se ha generado la documentación del manual de usuario para la instalación y uso del etiquetador y se ha creado una máquina virtual para facilitar el uso de la misma, ya que evita el tener que realizar la instalación, la cual puede ser algo confusa.
\subsection{Sprint 3}
Se ha realizado un script para comprobar la corrección de las coordenadas sobre las imágenes. El anterior etiquetador no etiquetaba bien y por lo tanto no ha habido aciertos.
\subsection{Sprint 4}
Se ha comenzado a investigar para realizar el despliegue de la aplicación. En un inicio se pensó en Heroku como servidor y Flask como framework y se han realizado pruebas para comprobar si es lo más adecuado. 
\subsection{Sprint 5}
Se ha decidido cambiar el despliegue de Heroku a Digita, Ocean con el uso de nanobox. En cuanto a los casos de uso del sistema, de momento, se va a usar la estructura empleada en la versión previa del proyecto.
\section{Estudio de viabilidad}

\subsection{Viabilidad económica}

\subsection{Viabilidad legal}


